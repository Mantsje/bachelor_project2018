\begin{abstract}
\addchaptertocentry{\abstractname} % Add the abstract to the table of contents
Developers all over the world use their favourite editors in order to creates software. One of the things all these editors tend to have in common is support for syntax highlighting. Syntax highlighting is the process of colouring certain words or sections of text in a document in order to give visual meaning to them. This is useful since humans can easily identify things by colour and the colours give an overall more structured look to an otherwise plain looking text document. Creating programs that do this highlighting for us is a development task in its own. The syntax highlighters tend to use (some form of) the underlying mathematical formalism of a programming language called the Context-Free-Grammar of this language. This is necessary in order to generate correct highlights. There are different forms of syntax-highlighters, some work with states and a simple stack much like finite state machines. Others, like the Rascal highlighter in Eclipse, use the Grammar directly.\\\\ 
In this thesis I will present an algorithm that takes in a CFG with minimal extra information. From this it will automatically generate a syntax-highlighter for state based syntax highlighters such as the ones used by Sublime-Text and TextMate. After a discussion of the algorithm itself I will dive into my implementation of this algorithm in Rascal.
\end{abstract}