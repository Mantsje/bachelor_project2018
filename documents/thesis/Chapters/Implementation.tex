% Chapter Template

\chapter{Rascal Implementation} % Main chapter title

\label{chap:Implementation} % Change X to a consecutive number; for referencing this chapter elsewhere, use \ref{ChapterX}

\section{Motivation for Rascal} \label{sec:RascalMotivation}
Rascal is a language workbench. It is also a language to be used for many high-level tasks involving the development of new languages and analysis on those and other languages. Since it is very high-level it makes it easier to tackle a rather difficult problem as this project is faced with. Doing something like this in a language like C would be near-impossible. Rascal provides a grammar formalism, algebraic datatypes and high-level operations on structures like graphs and sets.\\ 
The official website\footnote{source: \cite{website:Rascal}} states:
\begin{quotation}\textit{
"Rascal integrates source code analysis, transformation, and generation of primitives on the language level. It can be used for any kind of metaprogramming task: to construct parsers for programming languages, to analyze and transform source code, or to define new DSLs with full IDE support\ldots\\ Rascal primitives include immutable data, context-free grammars and algebraic data-types, relations, relational calculus operators, advanced pattern matching, generic type-safe traversal, comprehensions, concrete syntax for objects, lexically scoped backtracking, and string templates for code generation."
}\end{quotation}
The feature that is being developed by this project is not yet implemented in Rascal, as Rascal does syntax highlighting based on the $CFG$ inside the Eclipse IDE. Rascal itself is also not exported to the various editors with any proper highlighting. This together with all of the above are reasons why Rascal was chosen as the language to work with for this project.

\pagebreak
\section{Grammar formalism} \label{sec:RascalGrammar}
Rascal\footnote{source: \cite{website:RascalSyntax}} has a build in grammar formalism that can be used to develop grammars. Rascal will then generate a parser for this language and generate parse trees and abstract syntax trees for the grammar. One of the special things about Rascal is that it does not care about the complexity of the grammar. It does not need a grammar that is $LR(1)$, theoretically it can generate parsers for any $CFG$. However, there is often ambiguity found in a grammar, this can cause the parser-generator to fail. For this reason Rascal has something called disambiguation constructs. These are handles for the generator such that it can generate parsers without ambiguity. Since this algorithm operates on the grammar-level there is no need to dive into the parsing and evaluation that Rascal does. It is only important that all features that the grammar formalism uses are covered by the syntax-highlighter generator.

\subsection{Simple grammar definition}
Below is a simple grammar shown in Rascal. The keyword \gram{start} denotes the start-symbol $S$ of a grammar. There are four keywords that denote a non-terminal token. These are \gram{syntax, lexical, layout} and \gram{keyword}. These all have different meanings to the parser-generator. \gram{syntax}-tokens and their rules are internally interleaved with the \gram{layout}-token that is defined for the module the grammar is defined in (or if unspecified the $\$default\$$ layout which is empty):
\begin{center}
\gram{syntax S} \gram{= A B} 
\\internally becomes
\\\gram{syntax S} \gram{= A Layout B}
\end{center} 
\gram{lexical}-tokens are the same as \gram{syntax}, except for the fact that these do not get interleaved with \gram{layout}-tokens. The \gram{keyword}-tokens are a special kind of tokens that are used for disambiguation. It is used for making sure that keywords are not seen as identifiers for example. These tokens can only have single character classes or string literals as their rules. So no non-terminals nor regular expressions.
\lstinputlisting[language=RascalGrammar, caption={A Simple Example Grammar in Rascal}]{Code/grammars/simple/simple.grammar}

\pagebreak
\subsection{More advanced constructs}
	\subsubsection{Regular tokens}
	Rascal has a number of things which help the grammar-writer to develop new grammars quicker without the need to introduce new intermediate tokens to achieve the same result. These are explained and shown below.
		\begin{itemize}
			\item \emph{Character classes}\\
			The simple regular expression concept that defines a character class between square brackets. Negative classes are an option and Rascal supports everything including Unicode characters. Characters are internally represented as integer numbers.
\begin{lstlisting}[language=RascalGrammar]
lexical charclass 
	= [a-z]
	| ![%]
	;
\end{lstlisting}
			\item \emph{Star, Opt and Plus-operator}\\
			The familiar Kleene star and plus operation for zero or more and one or more repetitions respectively. Together with the question mark for zero or one occurrence.
\begin{lstlisting}[language=RascalGrammar]
lexical B_STAR = B*;
lexical B_PLUS = B+;
lexical B_OPT = B?;
\end{lstlisting}
			\item \emph{Begin- and End-of-Line}\\
			Also simple and familiar, the beginning and end-of-line requirements for a regular expression. Denoted with \gram{^} and \gram{\$}.
			\item \emph{Iterseps}\\
			A more special kind of token, also defined on Kleene star and plus with 0 or 1 or more. This is a structure that takes a main token and separator. This means that any list of the \gram{Main} tokens, interleaved with the list of separators, is accepted. This is very handy in for example parsing lists of parameters that are split by a comma.
\begin{lstlisting}[language=RascalGrammar]
lexical ParamList = {IdType ","}*;
/*parses for example: int a, float b, string c*/
\end{lstlisting}
		\end{itemize}
	\subsubsection{Labels}
	Labels are simple structures that allow naming of tokens inside a production rule, or to an entire production rule. Below are \gram{lhs, op, rhs} all labels given to tokens and is \gram{num} the name given to the entire rule with \gram{"42"}.
	\begin{lstlisting}[language=RascalGrammar]
	lexical Exp
		= Exp lhs "+" op Exp rhs
		| num: "42"
		;\end{lstlisting}
		
	\subsubsection{Tags}
	Tags are a special kind of annotation that can be added to production rules. They can give certain messages to the parser-generator. This can be messages like: this type of rule is a comment. Another message not intended for the generator could be: the tokens of this rule should receive a certain colouring. 
\begin{lstlisting}[language=RascalGrammar]
lexical Comment
	= @@Comment>@ "/*" ComChar "*/"
	;
	
lexical String
	= @@Context="string.quoted.doulbe null null null string.quoted.doulbe">@
	  StringBody "\<" Interpolated "\>" StringBody
	;\end{lstlisting}	
	\subsubsection{Disambiguation}
	Rascal has a number of handles to remove ambiguity from a grammar. This is useful for parser-generating, however not always for syntax-highlighting. Nevertheless the implementation should take all these factors into account. 
		\begin{itemize}
			\item \emph{Keywords}\\
			The first is the keywords construct that was mentioned in the previous section. It can be used to match a certain expression minus the tokens specified in a \gram{keyword}. The expression below matches all tokens consisting of the regex, except for those that match something in $Keywords$
			\begin{flushleft}
				\gram{keyword Keywords = "if" \| "else" \| "for" \| "while";}\\
				\gram{lexical Id = [a-zA-Z][a-zA-Z0-9]* \\ Keywords;}
			\end{flushleft}
			\item \emph{Precede and Follow restrictions}\\
			Because Rascal accepts all general $CFG$'s it does not implement longest or shortest match. It matches all possible variants. The only way to force rascal to take either longest and/or shortest match is through follow and precede-restrictions. In simple terms: \gram{[a-z]*} and input \gram{abc}, matches: $[\epsilon, a, b, c, ab, bc, abc]$. One can either enforce that something follows or precedes, but also the 'not' of those two.
			\begin{flushleft}
				\gram{input = 'abc'}
				\\\gram{lexical Id = [a-z]*;}						\hfill 	matches: $\{\epsilon, a, b, c, ab, bc, abc\}$
				\\\gram{lexical Id = [a-z]* !>> [a-z];} 			\hfill 	matches: $\{abc, bc, c\}$
				\\\gram{lexical Id = [a-z] !<< [a-z]*;} 			\hfill 	matches: $\{a, ab, abc\}$
				\\\gram{lexical Id = [a-z] !<< [a-z]* !>> [a-z];} 	\hfill 	matches: $\{abc\}$
			\end{flushleft}
			\item \emph{Associativity}\\
			Rules can receive associativity which reduces ambiguity in parsing. Think of a grammar like expression. Normally you have to write the grammar in a \gram{Exp = Expr "+" Term} way. Otherwise your parse-tree ends up being wrong on evaluation. However Rascal allows \gram{Exp = Exp "+" Exp}, together with the associativity ruling to influence a parse tree to nest correctly and prevent ambiguity. The website states:
\begin{flushleft}\textit{
Using Associativity declarations we may disambiguate binary recursive operators. The semantics are that an associativity modifier will instruct the parser to disallow certain productions to nest at particular argument positions:
\\\indent - Left and assoc will disallow productions to directly nest in their right-most position.
\\\indent - Right will disallow productions to directly nest in their left-most position.
\\\indent - Non-assoc will disallow productions to directly nest in either their left-most or their right-most position.
}\end{flushleft}
			\item \emph{Priority}\\
			Rules can receive priority over other rule(s) which reduces ambiguity in parsing. Think of a grammar like expression. Normally you have to write the grammar in terms of \gram{Exp, Term, Factor} way. Otherwise there is no way to express operator precedence. However in Rascal you can do the following: \gram{Exp = Exp "+" Exp, Exp = Exp "*" Exp}, together with the precedence rules you can parse and find correct parse trees. In the example below, the rule \gram{"42"} has precedence over the "times" operator and that has precedence over the plus.
\begin{lstlisting}[language=RascalGrammar]
lexical Exp
	= "42"
	> left Exp "*" Exp
	> left Exp "+" Exp
	;\end{lstlisting}		
	
	\item \emph{Except}\\
	Except rules are used in order to prevent certain labeled tokens or productions to be matched in a rule. This is done through putting an exclamation mark together with the to-be-excepted labelname at the end of a production rule.
	\begin{lstlisting}[language=RascalGrammar]
syntax S = String!illegal;

lexical String 
	= legal: "whatstringscanbe"
	| illegal: "whatstringscannotbe"
	;\end{lstlisting}
\end{itemize}

\subsubsection{An advanced example grammar}
In appendix \ref{app:Grammars} a grammar is shown that makes use of most of the more advanced constructs that were discussed in these past sections.

\pagebreak


\subsection{Internal representation}
Internally the grammars are represented a little differently. In short, all tokens (terminals and non-terminals), are of a datatype called Symbol. This is an algebraic datatype that can have symbols nested. A production rule also has a datatype called Production. This is also an algebraic datatype that shows a lot of the structure of the formal definition of a rule $p \in P \Rightarrow \prodgr{A}{w}$.
\subsubsection{Symbols}
Symbols are ADT's and divided into four sections seen below. The first one is the one to denote that some symbol is a start symbol. Line four through seven represent the tokens \gram{syntax, lexical, layout, keyword}. The parameterized versions of \gram{syntax} and \gram{lexical} are special versions of these definitions. They can be defined with a special datatype in order to generalize certain symbols. These are virtually nowhere to be found in any grammar used for this project and therefore not much attention was paid to these two symbols.\\ 
Following these lines there are three options defining terminal tokens. These are string-literals, case-insensitive literals and the character-classes.\\
Lines 15 through 22 represent the regular options for a token. Examples of these in order are: \gram{"", A?, A+, A*, \{A ","\}+, \{A ","\}*, (A\|B), (A B)}. These last two are used for alternatives and sequences that are inside a single production rule in parentheses.\\
The final line shows how Conditions are represented. These include all disambiguation constructs on Symbol-level and the begin- and end-of-line operators. The datatype of the Condition can be found in appendix \ref{app:RascalCondition}. 
\lstinputlisting{Code/datatypes/Symbol.data}
\subsubsection{Productions}
The production datatype is the type that represents elements of $P$ in a grammar. Below and in the appendices the datatype is shown. However there a number of unused rules when it comes to the algorithm. The \textit{Rascal Tutor} states:
\begin{itemize}
\item A prod is a rule of a grammar, with a defined non-terminal, a list of terminal and/or non-terminal symbols and a possibly empty set of attributes.
\item A regular is a regular expression, i.e. a repeated construct.
\item An error represents a parse error.
\item A skipped represents skipped input during error recovery.
\item priority means ordered choice, where alternatives are tried from left to right;
\item assoc means all alternatives are acceptable, but nested on the declared side;
\item reference means a reference to another production rule which should be substituted there, for extending priority chains and such.
\end{itemize}
From this it is clear to see that \data{error and skipped} are useless. Less obvious is the regular, one might think that this refers to the regular symbols that are possible. This however refers to the built-in Regex-engine that Rascal has and is therefore not part of this. Reference could be used in a grammar, however this was never encountered during any of the grammars that were tried. Recursive priorities are flattened to only a single priority. Every nested group of rules are rebuild to a \data{\\choice()} and put as a single item in the list of choices in the priority object.
\lstinputlisting{Code/datatypes/Production.data}

\pagebreak

\section{Implementing the Algorithm in Rascal} \label{sec:RascalAlgorithm}
The upcoming section discusses the implementation in Rascal. This includes datatypes, functions and algorithms, however no results or examples are shown. All examples generated from the implementation are results and will be discussed in the corresponding chapter.
\subsection{ToPlainGrammar} \label{sec:RascalFeatureToRemove}
The first step is to reduce the input to a plain grammar that was defined in \ref{sec:CFG_def}. Using all that was discussed above there are number of features that Rascal supports that are not of importance to the algorithm and need to be removed or rewritten. These occur at production- as well as symbol-level. A list of features to be removed or redone at the symbol-level include:
\begin{itemize}
\item labels, they will not be used in the highlighter and can be removed
\item tags that do not describe contexts can be removed
\item conditions, some of these are not important to a highlighter.
	\subitem \data{\\at-column(_)} state-based highlighters do not support a check like this.
	\subitem \data{\\except(_)}, labels were removed so this is not useful anymore either.
\item All regular-symbols can be rewritten to terminals, non-terminals and production rules.
	\begin{lstlisting}[language=RascalGrammar]
	A?				-> A_OPT 								= A | ;
	A+				-> A_PLUS 							= A | A A_PLUS;
	A*				-> A_STAR 							=   | A A_STAR;
	(A | B)		-> A_B_ALT 							= A | B;
	(A B)			-> A_B_SEQ 							= A B;
	{A ","}+	-> A_,_ITERSEPS 				= A A_,_ITERSEPSTAIL
						-> A_,_ITERSEPSTAIL 		= "," A A_,_ITERSEPSTAIL | ;
	{A ","}*	-> A_,_ITERSTARSEPS 		= A A_,_ITERSTARSEPSTAIL | ;
						-> A_,_ITERSTARSEPSTAIL = "," A A_,_ITERSEPSTAIL | ;\end{lstlisting}
\end{itemize}
The Sublime syntax-highlighters support character classes as well and therefore it is unnecessary to remove these. The other conditionals can be represented with a special regular expression using lookahead and lookbehind. Also all tokens that have nested symbols being a terminal are not rewritten to new Symbols but kept as regular expressions. So \gram{[a-z]+} is kept as is and does not become. \\ \gram{[a-z]_PLUS = [a-z] [a-z]_PLUS \| [a-z];}\\\\
The best approximation of a formal production rule $p \in P$ is the\\ 
\data{\\prod(Symbol lhs, list[Symbol] rhs, set[Attr] attributes)}.\\
Of the productions that can be encountered there are 3 in conflict: \data{priority},\\ \data{associativity} and \data{choice}. Choice only holds a set of Productions together with the left-hand symbol that these have in common. If all elements of choice are of type \data{prod}, then choice can remain for ease of use. A highlighter does not produce parse-trees so the associativity rules are not important and can therefore be removed. This is done through extracting the set of alternatives from the Production and removing the associativity attributes from the rules found in the set.\\
For priority there are two options. Either rewrite the rules to keep the structure, or just remove them as done with associativity. The first option is like rewriting\\
\gram{Exp = "42" > Exp "*" Exp > Exp "+" Exp;}\\
to\\
\gram{Exp = Exp "+" Exp \| Exp1; Exp1 = Exp1 "*" Exp1 \| Exp2; Exp2 = "42";}\\
The second option is just like with associativity and flatten priority to a choice Production. I have implemented both versions which will be discussed in the \nameref{chap:results} chapter. The nice thing about rewriting it in this way is that you can keep using the grammar formalism of Rascal and all its features, like writing an internal representation back to the human representation.

\subsubsection{Layouts}
I chose to remove all layouts from all of the production rules. This is due to the fact that generally the \gram{layout}-tokens are used for 2 things: whitespace and comments. Although comments are important to the highlighter, whitespace is not. I chose to extract them from the productions but keep them behind such that these can go into a prototype context\footnote{prototype contexts were contexts that are always included (\ref{sec:SublimeSyntax})}. This way the rest of the machines become way less cluttered with possible whitespace parsing.

\subsubsection{Scope information extraction}
The preserving of the scopes to be assigned are handled in the beginning of the process. Every symbol of the simplified grammar is mapped to the corresponding scope that is saved for later use. However there is a limitation in the current implementation which causes loss of power in the potential highlighter. This is the fact that any token can only have one scope assigned to it over the stretch of the entire grammar. This is true for non-terminals as well as terminals (so no different colouring between function call, definition or variable declaration identifiers). If nested conflicts arise then the deepest scope is assigned to the matched token. (E.G. an entire machine of some non-terminal wants scope A and a Token inside this non-terminal's rules has scope B). Due to a lack of time and performance of the algorithm this was implemented in this way and not extended to the better variant described in earlier chapters.

\pagebreak

\subsubsection{The implementation}
The runtime of this part of code is linear with the size of the input grammar, where the size of a grammar is defined by the number of production-rules together with the total number of Symbols present in the grammar (including nested ones). This can be seen in the function below. First it rewrites all rules of a grammar to a set of plain elements of the type \data{\\prod(Symbol lh, list[Symbol] rhs, set[Attr] attributes)}. After this, all symbols in the right-hand side of a rule are rewritten in the way that was just described. Then, as a final action, based on the passed parameter, remove all the interleaved \gram{layout}-tokens. However, rules corresponding to layouts themselves are kept. This results in a new grammar in which by normal means the \gram{layout}-symbols are unreachable, however this is fixed through adding them as a prototype context later.
\lstinputlisting[language=Rascal]{Code/ToPlainGrammar.rsc}


\subsection{Grammar2Graph}
There was already something present in Rascal that created a graph with dependencies of symbols. This however did not search recursively enough, so I wrote my own variant that dives recursively into a symbol and finds any general non-terminal token that is not inside a \data{Condition}. The build-in Graph datatype is nice and generically written, however the function to compute its strongly connected components was not yet implemented. So I implemented Kosaraju's algorithm in order to compute the components.\\\\
Both the \emph{grammar2graph} and \emph{Kosaraju's Algorithm} are shown below. The functioning of \textit{grammar2graph} can be written as: For each \data{Symbol elem} in the right hand side of any rule in the grammar, check whether the left-hand side $s$ is a label with a plain symbol or a plain Symbol itself. Assign the name $from$ to the plain symbol. Now if $from$ is some instance of a non-terminal, enter the second for-loop. For each symbol that is a plain non-terminal token inside $elem$ (including $elem$ itself), add an edge from the symbol $from$ to the plain non-terminal that was found.\\
One extra feature was added to this function. Every vertex gets a self-loop. This is because of Rascal's graph implementation being just the set of edges. This disallows disconnected vertices to be part of a graph. To circumvent this, I added this little line that is not entirely correct, but helps Kosaraju's algorithm to find all components, including singular components with no edges.\pagebreak
\lstinputlisting[language=Rascal]{Code/Grammar2Graph.rsc}

\subsubsection{Kosaraju}
Below is the algorithm of Kosaraju shown. $components$ is a map that maps every edge to its corresponding component. Returning the range of this map is the set of strongly connected components. It is written generically so it can be used on any Rascal-defined Graph.
\lstinputlisting[language=Rascal]{Code/Kosaraju.rsc}

\pagebreak
\subsection{ToStronglyRegularGrammar}
The transformation described in \cite{MohriNederhof} is implemented in rascal. It handles all symbols except parameterized lexicals and parameterized syntax-tokens. These throw exceptions everywhere they cause a problem, so this is easy to extend. These were never encountered so not really in scope for this project. The conversion is preceded by performing the $ToPlainGrammar$-function, since it expects a grammar consisting of only \data{\\prod(_,_,_) and \\choice(_,_)}. As an intended side-effect this transformation removes all attributes from the grammar. This makes the new grammar loose all information on contexts and scopes since this should be extracted from the plain version of the grammar and not the strongly regular one. Below is the \emph{strongly regular} version of the expression-grammar. The $A'$s are replaced with $A\_end$, since the primed versions of the tokens correspond to parsing the ending of a token.

\subsection{Converting the grammar to StateMachines} 
After the grammar is rewritten to a strongly regular one it needs to be rewritten to a set of statemachines. Rascal has a datastructure callled \data{map[&T from, &S to]}. This can be used to map a single Symbol or set of Symbols (I.E. a component) to a StateMachine. Using the algorithm from \ref{sec:ComponentMachine} the machines can first be generated per component and then mapped to the Symbols which are then mapped to $DFA$'s. This is done through the standard powerset construction. So no NFA's per component because the next step is to convert them to DFA's. The implementation is slightly different from the one described, because it introduces a few extra states. Instead of introducing one transition $(q_S,\ a,\ q_B)$ for the rule \data{S = "a" B}, there are two rules introduced: $(q_S,\ a,\ q_{term\_a})$ and $(q_{term\_a},\ \epsilon,\ q_B)$. These intermediate states are all uniquely named. Therefore, not every instance of a terminal $a$ goes to the, same $q_{term\_a}$.

\pagebreak
\subsubsection{StateMachine datatype} \label{sec:RascalStatemachine}
In the section below\footnote{In the actual code $delta$, $q0$ and $F$ are replaced with $transitions$, $startState$ and $finalStates$.} the datatype of a StateMachine is shown. States are represented by strings, a Token is something that can be on an arc, which in this case is a Symbol. A Transition is a 3-element tuple that holds two states, from and where, and a Token that triggers the transition. A DeltaFunction is a set of Transitions that can be indexed as an array. For example: Transition tr[$s_0$] returns a set of 2-element tuples of type \data{set[tuple[Token, State]]}, which are all possible combinations of tokens and resulting state that $s_0$ has in the machine. Now a statemachine is either an $NFA$ or a $DFA$. This is decided with the constructor that is chosen on instantiation. There is no actual prevention of adding non-determinism to a $DFA$. This is left to the programmer. For $\epsilon$-transitions I created an extension to Rascal's Symbol called \data{\\eps()}. As can be observed the states are not saved as separate items in the machine since they can be obtained from $\delta$.
\lstinputlisting[basicstyle=\ttfamily\small]{Code/datatypes/StateMachine.data}

\subsubsection{Conflict resolution} \label{sec:RascalConflicts}
Resolving the conflicts is done as described in \ref{sec:ConflictResolutionAlgorithm}. This is done is almost the exact way the algorithms describe. 

\subsection{Mapping to Contexts} \label{sec:RascalToContexts}
The mapping to contexts is largely done through the description in chapter \ref{chap:Algorithm}. Rascal has a few extra features that are left to consider, being the Conditional Symbols. These are harder to process since the conditions themselves can result in a prefix or a postfix of a regular expression. For example: \gram{[a-z] !<< [a-z]+ !>> [a-z]} can be expressed with lookahead and -behind as \data{(?<![a-z])([a-z]+)(?![a-z])}. Luckily they are all expressable in the regex engine that Sublime and other editors like TextMate use. Precede and follow restrictions translate into lookbehind and lookahead. The \data{\\delete(_)} becomes negative lokahead before the actual match (E.G. \data{id \ "for"} result in match: \data{'(?!for)(id)'}). The begin and end-of-line are also implemented in the regex engines as the same tokens. So except for doing a few tricks with generating correct regular expressions this step is done in the same manner as was described in the previous chapter.

\pagebreak
\subsubsection{Highlighter datatype} \label{sec:RascalHighlighter}
Below is the datatype that represents a state-based syntax highlighter in Rascal. It represents much of the structure of a Sublime Highlighter, but with minimal extensions or modifications other editors could be added. There is also a function that writes an instance of this type to a \emph{.sublime-syntax} file. An $SHRegex$ is regular expression to be set as a match. The SyntaxHighlighter is the top level object. It contains a set of $SHVar$ which are named regular expressions that can be defined for reuse. The rest of the constructors and types speak for themselves.
\lstinputlisting{Code/datatypes/Highlighter.data}