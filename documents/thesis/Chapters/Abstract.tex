\begin{abstract}
\addchaptertocentry{\abstractname} % Add the abstract to the table of contents
Developers all over the world use their favourite editors in order to create software. One of the things all these editors tend to have in common is support for syntax highlighting. Syntax highlighting is the process of colouring certain words or sections of text in a document in order to give visual meaning to them. Some highlighters use the context-free grammar or parse trees of a language, however state-based highlihgters are simpler and cannot handle the complex structures of the aforementioned entities. Hence, a translation step from $CFG$ into a state-based highlighter is needed. \\\\ 
In this thesis I will present an algorithm that takes in a CFG with minimal extra information. From this it will automatically generate a syntax-highlighter for state based syntax highlighters such as the ones used by Sublime-Text and TextMate. After a discussion of the algorithm itself I will dive into my implementation of this algorithm in Rascal.
The resulting algorithm generates negative results for the problem that is faced. A multitude of problems surface when taking the described approach. These errors are so severe that there is no fixing nor circumventing them. This thesis will highlight these errors, where they originate and how they could be avoided in the future.
\end{abstract}